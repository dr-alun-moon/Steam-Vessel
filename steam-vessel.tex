%&pdflatex
\documentclass[12pt,a4paper]{article}
\usepackage{siunitx}
\usepackage{array}
\usepackage{booktabs}
\begin{document}
\section{Some equations}
\paragraph{Flow through opening} goes with square-root of the pressure
\begin{equation}
	q_m = \rho q_v = C A_2 \sqrt{2\rho(p_1-p_2)}
\end{equation}
Where $C$ is the flow coefficient, $A_2$ is the area of the opening, 
and $p_1$ the pressure in the vessel with $p_2$ the external pressure.

\begin{table}
\begin{tabular}{l>{$}c<{$}Ss}
	\toprule
	latent heat of fusion & Q_m & 334 & \kilo\joule\per\kilogram\\
	melting point & T_m & 0 & \degreeCelsius\\
	latent heat of vaporization & Q_v & 2264.705 & \kilo\joule\per\kilogram\\
	boiling point & T_v & 100 & \degreeCelsius\\
	Heat Capacity at \SI{25}{\degreeCelsius} & C_{25} & 4.1813 & \joule\per\gram\per\kelvin\\
	Heat capacity at \SI{100}{\degreeCelsius} (steam) && 2.080 & \joule\per\gram\per\kelvin\\
	Desity of steam at \SI{100}{\degreeCelsius} and 1atm pressure && 0.6 & 
		\kilogram\per\meter\cubed\\
	\bottomrule
\end{tabular}
\end{table}

\paragraph{Heat Capacity}
\begin{equation}
	\Delta T = \frac{\Delta H}{Cm}
\end{equation}
or for a mixture
\begin{equation}
	\Delta T = \frac{\Delta F}{C_1 m_1 + C_2 m_2}
\end{equation}

\end{document}
